\section{Einleitung}
In einer vernetzen Welt wie unserer, werden unabl�ssig neue und bessere Web-Technologien entwickelt. Diese neuen Technologien bringen zum einen eine bessere Programmierfreundlichkeit mit sich, aber sie sind zum anderen durch neue Ans�tze auch schneller als �ltere Technologien.

Um heute eine Webanwendung zu entwickeln die nicht nur tut was sie soll, sondern die auch performat und auf vielen Systemen l�uft, muss eine Vielzahl von Technoliegen beherrscht und angewendet werden.
%kreterien mobile endger�te betriebssystem browser asynchron (Website als Anwendung)
Das Ziel diese Arbeit ist es herauszufinden wie neue Technoliegen von heute kombiniert werden k�nnen, um ein m�glichst leistungsstarkes und performantes System zu schaffen.
%moderne aktuelle (nicht heute)
Hierbei sollen Programmierans�tze wie Google Polymer, AngularJS, HTML5, CSS3, PHP7 und Google Dart unter dem Portalserver Typo3 vereint werden.
%typo weg
%gradwanderung zwischen performant und leistungsstark

Es soll gepr�ft werden, wie und ob es m�glich ist diese verschiedenen Technoliegen in m�glichst modularen Typo3-Extensions unterzubringen.
%typo weg

Diese Arbeit soll der theoretischen Grundstock f�r eine weitere Arbeit sein, in der das Support-Portal der PDV System Erfurt GmbH neu entwickelt wird.
Im Verlauf sollen mehrer Beispiel Extensions f�r Typo3 entstehen, welche das Ziel verfolgen Programmierans�tze f�r eine sp�tere Neuentwicklung zu sein.
%typo weg

In den nun folgenden Abschnitten werden Progammierbeispiele und Hinweise gegeben, wie eine solche Neuentwicklung unter den Gesichtspunkten Performance, Umsetzbarkeit und Usability vorgenommen werden kann.

\section{Portalserver/ CMS-Systeme im Vergleich}
\subsection{Typo3}
\subsection{Typo3 Neos}
\subsection{Joomla}
\subsection{Drupal}
\subsection{Auswertung der m�glichkeiten}
\section{Typo3}
\subsection{Typo3 8.2}
\subsection{Extensios}
\subsection{TypoScript}
\subsection{Fluid}
\section{Modulare Extensions}
\section{Neue Webtechnologien}
\subsection{Google Polymer}
\subsection{AngularJS}
\subsection{HTML5}
\subsection{CSS3}
\subsubsection{Bootstrap}
\subsubsection{Foundation Framework}
\subsection{PHP7}
\subsection{Google Dart}
\section{Zusammenspiel der Technologien}
\section{Zusammenfassung}
\section{Fazit}