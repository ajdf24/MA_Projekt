\section{Einleitung}
In einer vernetzten Welt wie unserer werden unabl�ssig neue und bessere Web-Technologien entwickelt. Diese neuen Technologien bringen eine bessere Programmierfreundlichkeit mit sich, sind aber zum anderen durch neue Ans�tze auch schneller als �ltere Technologien.

Moderne Webanwendungen m�ssen eine Vielzahl an Kriterien erf�llen. Eine gute Webseite sollte heute m�glichst auf einem mobilen Endger�t, sowie auf einem Desktop-PC oder Smart-TV darstellbar sein. Um dies zu erm�glichen, werden besondere Anforderungen an das Design gestellt, wobei auch m�glichst alle g�ngigen Browser unterst�tzt werden m�ssen.
Ein weiterer Punkt ist, dass Webseiten sich heute so fl�ssig und performant wie ein natives Programm verhalten sollen. Hierbei ist es zudem wichtig, vorhandene Hardware anzusprechen und mit ihr zu interagieren.

Das vorliegende Projekt soll den theoretischen Grundstock f�r eine weitere Arbeit bilden, in der das Support-Portal der PDV System Erfurt GmbH neu entwickelt wird.
Dabei, soll evaluiert werden, welcher der g�ngigen Portalserver f�r ein Supportportal am besten geeignet ist.
Au�erdem soll untersucht, wie es m�glich ist, modulare Erweiterungen f�r einen solchen Server zu entwickeln, um unn�tige Datenredundanz zu vermeiden. 

Es soll zudem untersucht werden, wie und ob es m�glich ist, diese verschiedenen Technologien in m�glichst modularen Erweiterungen unterzubringen, um eine m�glichst performante und zu gleich komplexe und umfangreiche Webseite zu entwickeln. Durch die Verwendung neuer Technologien soll au�erdem eine m�glichst gro�e Funktionsvielfalt erreicht werden.
Dieses Oxymoron aufzul�sen, ist der Schwerpunkt der vorliegenden Arbeit, obwohl doch schon jetzt ersichtlich ist, dass f�r beide Seiten ein Kompromiss gefunden werden muss. Wie dieser aussehen kann, wird im weiteren Verlauf eine wichtige Rolle spielen.
%typo weg

Ziel dieser Arbeit ist es, herauszufinden, wie unter zu Hilfenahme moderner Technologien alle diese Kriterien m�glichst gut erf�llt werden k�nnen.
Es sollen Programmierans�tze wie Google Polymer, AngularJS, HTML5, CSS3, PHP7 und Google Dart unter dem Portalserver vereint werden, welcher die Webseite beziehungsweise die Anwendung bereitstellt.

Durch einen modularen Aufbau, sollen au�erdem die Schnittstellen zwischen verschieden Erweiterungen vereinfacht werden. Zus�tzlich soll es m�glich werden, nur bestimmte Erweiterungen zu verwenden, wenn f�r den jeweiligen Use-Case nicht alle Erweiterungen ben�tigt werden.


% Im Verlauf sollen mehrer Beispiel Extensions f�r einen Portalserver entstehen, welche das Ziel verfolgen Programmierans�tze f�r eine sp�tere Neuentwicklung zu sein.
%typo weg

In den nun folgenden Abschnitten werden Progammierbeispiele und Hinweise gegeben, wie eine solche Neuentwicklung unter den Gesichtspunkten Performance, Umsetzbarkeit und Usability vorgenommen werden kann, um Mitarbeitern und Kunden ein m�glichst performantes Supportportal bieten zu k�nnen. Dieses System soll den Arbeitsalltag eines jeden Nutzers erleichert.

