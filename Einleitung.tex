\section{Einleitung}
In einer vernetzen Welt wie unserer, werden unabl�ssig neue und bessere Web-Technologien entwickelt. Diese neuen Technologien bringen zum einen eine bessere Programmierfreundlichkeit mit sich, aber sie sind zum anderen auch Performanter als fr�here Ans�tze.

Um heute eine Webanwendung zu entwickeln die nicht nur tut was sie soll, sondern die auch performat und auf vielen Systemen l�uft, muss eine Vielzahl von Technoliegen beherrscht und angewendet werden.

Das Ziel diese Arbeit ist es herauszufinden wie neue Technoliegen von heute kombiniert werden k�nnen, um ein m�glichst leistungsstarkes und performantes System zu schaffen.
Hierbei sollen Programmierans�tze wie Google Polymer, AngularJS, HTML5, CSS3, PHP7 und Google Dart under dem Portalserver Typo3 vereint werden.

Es soll gepr�ft werden, wie und ob es m�glich ist diese verschiedenen Technoliegen in m�glichst modularen Typo3-Extensions unterzubringen.

Diese Arbeit soll der theoretischen Grundstock f�r eine weitere Arbeit sein, in der das Support-Portal der PDV System Erfurt GmbH neu entwickelt wird.
Im Verlauf sollen mehrer Beispiel Extensions f�r Typo3 entstehen, welche das Ziel verfolgen Programmierans�tze f�r eine sp�tere Neuentwicklung zu sein.

In den nun folgenden Abschnitten werden Progammierbeispiele und Hinweise gegeben, wie eine solche Neuentwicklung unter den Gesichtspunkten Performance, Umsetzbarkeit und Usability vorgenommen werden kann.