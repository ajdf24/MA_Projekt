\section{Einleitung}
In einer vernetzen Welt wie unserer, werden unabl�ssig neue und bessere Web-Technologien entwickelt. Diese neuen Technologien bringen zum einen eine bessere Programmierfreundlichkeit mit sich, aber sie sind zum anderen durch neue Ans�tze auch schneller als �ltere Technologien.

Heutige Webanwendungen m�ssen eine Vielzahl an kriterien erf�llen. Eine gute Webseite sollte heute m�glichst auf einen mobilen Endger�t, sowie auf einem Desktop-PC oder Smart-TV darstellbar sein. Um dies zu erm�glichen, m�ssen besondere Anforderungen an das Design gestellt werden. 
Nat�rlich m�ssen sie auch von m�glichst allen Browsern anzeigbar sein.
Ein weitere Punkt ist, dass Webseiten sich heute so fl�ssig und performant wie ein natives Programm verhalten sollen. Hierbei ist es nat�rlich auch wichtig vorhandene Hardware anzusprechen und mit ihr zu interagieren.

Diese Arbeit soll der theoretischen Grundstock f�r eine weitere Arbeit sein, in der das Support-Portal der PDV System Erfurt GmbH neu entwickelt wird.
Im Verlauf dieser Arbeit, wird evaluiert werden, welcher der g�ngigen Portalserver f�r ein Supportportal am besten geeignet ist.
Au�erdem wird gepr�ft werden, wie es m�glich ist, modulare Erweiterungen f�r einen solchen Server zu entwickeln, um unn�tige Datenredundanz zu vermeiden. 

Es soll gepr�ft werden, wie und ob es m�glich ist diese verschiedenen Technologien in m�glichst modularen Erweiterungen unterzubringen, um eine m�glichst performate und zu gleich leistungsstarke Webseite zu entwickeln.
Dieses Oxymoron aufzul�sen ist der Schwerpunkt der vorliegenden Arbeit, obwohl doch schon jetzt ersichtlich ist, dass ein auf beiden Seiten ein Kompromiss gefunden werden muss. Wie dieser genau aussieht, wird im weiteren Verlauf eine wichtige Rolle spielen.
%typo weg

Ziel dieser Arbeit ist es herauszufinden, wie unter zu Hilfenahme moderner Technologien alle diese Kriterien m�glichst gut erf�llt werden k�nnen.
Es sollen Programmierans�tze wie Google Polymer, AngularJS, HTML5, CSS3, PHP7 und Google Dart unter dem Portalserver vereint werden, welcher die Webseite beziehungsweise die Anwendung bereitstellt.

Durch einen modularen Aufbau, soll au�erdem die Schnittstellen zwischen verschieden Erweiterungen vereinfacht werden. Des weiteren soll es m�glich werden, nur beliebige Erweiterungen zu verwenden, wenn f�r den jeweiligen Use-Case nicht alle ben�tigt werden.


% Im Verlauf sollen mehrer Beispiel Extensions f�r einen Portalserver entstehen, welche das Ziel verfolgen Programmierans�tze f�r eine sp�tere Neuentwicklung zu sein.
%typo weg

In den nun folgenden Abschnitten werden Progammierbeispiele und Hinweise gegeben, wie eine solche Neuentwicklung unter den Gesichtspunkten Performance, Umsetzbarkeit und Usability vorgenommen werden kann. 
Um Mitarbeitern und Kunden ein m�glichst performantes Supportportal bieten zu k�nnen, welches den Arbeitsalltag eines jeden Nutzers erleichert.

