\section{Modulare Extensions}\label{Modulare Extensions}
Im im Folgenden soll evaluiert werden in wie weit es m�glich ist, dass verschiedene Extensions zusammen arbeiten k�nnen.
Das Supportportal im Ganzen soll aus verschiedenen Extensions bestehen, welche modular auf einander aufbauen. 

Es sollen folgende Extensions erstellt werden:

\FloatBarrier
\begin{minipage}{3cm}
Template
\end{minipage}
\begin{minipage}{13cm}
In der Template Extension wird das grundlegende Layout des Supportportals definiert.\\
Zus�tzlich kommen noch einige ViewHelper und Utility-Funktion hinzu, welche von mehreren anderen Extensions genutzt werden k�nnen.\\
Alle anderen Extensions bauen auf dieser auf und sind ohne diese nicht Lauff�hig.\\
\end{minipage}

\begin{minipage}{3cm}
SupportBase
\end{minipage}
\begin{minipage}{13cm}
Die Extension SupportBase stellt das grundlegende Datenmodell zur Verf�gung, das andere Extensions nutzen und auch erweitern k�nnen.\\
Zus�tzlich k�mmert sich diese Extension auch um das Login der Nutzer. Dies schlie�t die Neuanmeldung und das �ndern des Passwortes mit ein.\\
\end{minipage}

\begin{minipage}{3cm}
Polymer
\end{minipage}
\begin{minipage}{13cm}
Die Polymer-Extension stellt Polymer-Elemente (siehe Abschnitt \ref{Typo und Polymer}) in Form von ViewHelpern (siehe Abschnitt \ref{Fluid}) zur Verf�gung. Sie bildet eine Sammlung von Polymer-Elementen, welche andere Extensions verwenden k�nnen.\\
Alle Extensions, welche Polymer-Elemente verwenden, sind von ihr abh�ngig.\\
\end{minipage}

\begin{minipage}{3cm}
DownloadPortal
\end{minipage}
\begin{minipage}{13cm}
Die Extension erweitert das Datenmodell der SupportBase und stellt f�r die Nutzer von PDV-Produkten wichtige Downloads bereit.\\
Zus�tzlich bietet sie eine Abo-Funkion an, in der Nutzer sich f�r bestimmte Download-Kategorien anmelden k�nnen. Bei neuen Downloads werden sie dann entsprechend benachrichtigt.\\
\end{minipage}

\begin{minipage}{3cm}
SupportPortal
\end{minipage}
\begin{minipage}{13cm}
Die SupportPortal-Extension ist das Herzst�ck des allumfassenden Supportportals. Mit ihr ist es m�glich Calls zu Problemen oder Software-Bugs zu �ffenen und mit Supportmitarbeitern in Kontakt zu treten.\\
\end{minipage}

\begin{minipage}{3cm}
KnowledgeBase
\end{minipage}
\begin{minipage}{13cm}
Die KnowledgeBase ist, wie der Name schon sagt, ein Verzeichnis mit bekannten Problemen und Hinweisen zum verwenden der PDV-Software. Sie bezieht sich auf Daten aus der SupportBase.\\
\end{minipage}

\begin{minipage}{3cm}
DataExchange
\end{minipage}
\begin{minipage}{13cm}
Um den Datenaustaus zwischen Kunden zu gew�hrleisten, soll die Extension DataExchange geschaffen werden. �ber sie ist es m�glich, dass PDV-Mitarbeiter Kunden Daten zur Verf�gung stellen und umgekehrt.\\
\end{minipage}

\begin{minipage}{3cm}
AdminSchulung
\end{minipage}
\begin{minipage}{13cm}
Die AdminSchulung ist eine Extension, welche es der PDV Systeme GmbH erlaubt Schulungen und Pr�fungen f�r Systemadministratoren online durchf�hren zu k�nnen.\\
\end{minipage}

Die Extensions Template, SupportBase und Polymer stellen grundlegende Extensions dar, von denen alle Support-Extensions abh�ngig sind. Die Support-Extensions DownloadPortal, SupportPortal, KnowledgeBase, DataExchange und AdminSchulung sind alle voneinander unabh�nig und k�nnen modular eingesetzt werden.

Innerhalb der vorliegenden Arbeit sollen nicht alle Extensions programmiert werden, da dies aus Gr�nden des Aufwandes nicht m�glich ist. Es soll lediglich Anhand der Extensions Template, Polymer, SupportBase und DownloadPortal gezeigt werden, dass modulare Extensions unter Typo3 m�glich sind und wie ein L�sungsansatz aussehen kann.

Im weiteren Verlauf wird ebenso auf die genaue Programmierung der einzelnen Extensions verzichtet. Es wird sich auf die Schwerpunkte bezogen, welche wichtig sind um die Extensions modular aufzubauen und wie es m�glich ist, moderne Web-Technologien wie Polymer oder AngularJS in Typo3 Extensions einzubauen (siehe Kapitel \ref{Zusammenspiel der Technologien}).

\subsection{Datenbasis}
\subsection{Rendering}