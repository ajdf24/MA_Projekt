\section{Neue Webtechnologien}
\subsection{Google Polymer}
\subsection{AngularJS}
\subsection{HTML5}
\subsection{CSS3}
\ac{CSS} ist eine Designsprache f�r HTML und ist deshalb eine der Hauptkomponenten der Webentwicklung. Mit Hilfe von \ac{CSS} ist es m�glich einzelne HTML-Elemente oder Gruppen zu stylen.
Seit der ersten Version die 1993 erschien, wird CSS kontinuirlich weiterentwickelt und ist heute auf praktisch jeder Web-Seite im einsatz.

Auf der Basis von CSS entwickelten sich im laufe der Jahre immer mehr Frameworks, welche einen Designansatz umsetzten und fertige Klassen f�r die Verwendung bereitstellen.
Eines der heute am h�ufigsten anzutreffenden \ac{CSS}-Frameworks ist "`Bootstrap"'.

Innerhalb der PDV Systeme Erfurt wurde beschlossen das neue Supportportal im "`Material Design"' aufzubauen. Als "`Material Design"' werden Gestaltungsrichtlinen von Google bezeichnet, welche angeben wie eine Android-Applikation oder eine mobile Web-Seite f�r Android aussehen sollte.

Die nun folgenden Betrachtungen beziehen sich auf die Umsetzung im "`Material Design"'.
\cite{}
\subsubsection{Bootstrap}
Bootstrap ist ein \ac{CSS} Framework, welches von "`Twitter"' entwickelt wird und das unter der "`MIT-Lizenz"' frei erh�ltlich ist. Durch den modularen Ansatz von Bootstrap ist es sehr leicht m�glich das Framework um eigene Style-Anweisungen zu erg�nzen.

Ein Nachteil von Bootstrap ist, dass es nicht leichtgewichtig ist. Zwar kann es sehr gut erweitert werden, dies macht jedoch das Framework auch Schwerf�llig. 
Soll zum Beispiel ein Bootstrap-Template f�r das "`Material Design"' verwendet werden, muss zun�chst das "`standard Framework"' eingebunden werden. Ein Template wie "`Material Design for Bootstrap"' �berschreibt dann nach der Einbindung zum Teil das standard Framework und erg�nzt es um eigene Klassen.
Der Vorteil bei diesem Vorgehen ist ganz klar, dass ein bestehendes Template relativ leicht ausgetauscht werden kann, ohne dass der eigentliche HTML-Code der Web-Seite angepasst werden muss.

Dieser Vorteil wird schnell zum Nachteil, wenn eine Web-Seite erstellt werden soll bei der es auf Geschwindigkeit ankommt. Durch das �berscheiben des standard Frameworks wird zus�tzliche Zeit beim parsen der Web-Seite ben�tigt, was vorallem bei �lteren PCs auff�llt. 
Zus�tzlich steigt der Overhead beim laden einer Seite, da mehr CSS geladen wird, als tats�chlich ben�tigt wird.

Standardm��ig enth�lt Bootstrap zum Beispiel keinen Date-Picker, welcher jedoch in einem Supportportal unerl�sslich ist. Auch "`Bootstrap Material"' enth�lt keinen Date-Picker im "`Material Design"'.
Durch den modularen Aufbau von Bootstrap kann nat�rlich schnell und einfach ein entsprechendendes Template erg�nzt werden. Dies bedeutet aber wieder zus�tzlichen Overhead.
\subsubsection{MaterializeCSS}
Ein anderes Framework, welches die Design-Richtlinien des "`Material Design"' umsetzt ist das noch recht neue Framework "`MaterializeCSS"'. 
MaterializeCSS ist ein eigenst�ndiges Framework, welches versucht alle gegebenen Gestaltungsrichtlinen m�glichst elegant und performat umzusetzen.

Die MaterializeCSS Quellen sind mit \ac{SASS} geschrieben, und m�ssen vor der Verwendung in CSS komiliert werden. Dieses Vorgehen hat meherer Vorteile. So ist es zum einen sehr einfach m�glich das Framework um eigene Style-Objete zu erweitern. Zum anderen hat es den Vorteil, dass die Farbgebung des gesamten Frameworks an einer Stelle zusammen gefasst ist. 
Soll also die Farbgebung ge�ndert werden, so werden in den \ac{SASS}-Quellen des Frameworks die Farben zentral angepasst. Nach einer erneuten kompilierung stehen dann die Farben im gesamten Framework zur Verf�gung.
Ohne dieses Vorgehen m�ssten die entsprechenden Farben an unz�hligen Stellen innerhalb der \ac{CSS}-Datei angepasst werden.

Ein weitere Vorteil von MaterializeCSS ist, das viele Templates wie ein Date-Picker im "`Material Design"' schon vorhanden sind. Diese Templates k�nnen ohne zus�tzlichen Overhead verwendet werden.

Auch wenn MaterializeCSS noch sehr jung im Vergleich zu Bootstrap ist, so ist es doch eine gute alternative, wenn eine Web-Seite im "`Material Design"' umgesetzt werden soll.
\subsection{PHP7}
\subsection{Google Dart}
\section{Zusammenspiel der Technologien}
\section{Zusammenfassung}
\section{Fazit}