\section{Portalserver/ CMS-Systeme im Vergleich}
Das zuk�nftige Supportportal der PDV Systeme GmbH soll auf Basis eines Portalservers bzw. \ac{CMS}-Servers aufgebaut werden. Hierf�r werden im folgenden einige M�glichkeiten genauer betrachtet.

Ein Portalserver, welcher f�r das Projekt heran gezogen wird muss die folgenden Eigenschaften aufweisen.
\begin{itemize}
 \item Webseiten m�ssen frei gestalltbar sein
 \item Es muss die M�glichkeit bestehen Anwendungen f�r den Server zu entwickeln
 \item Das System muss Open-Source sein, um ggf. in den Quellcode eingreifen zu k�nnen
 \item Die Nutzer-Community sollte m�glichst gro� sein, damit Probleme leicht diskutiert und behoben werden k�nnen
 \item Der Server muss die M�glichkeit bieten Dateien zu verwalten, welche als Download oder Kontent in das Portal einflie�en
\end{itemize}

Auf die Betrachtung reiner \ac{CMS}-System wird an dieser Stelle verzichtet, da diese nicht die gew�nschten Anforderungen eines Portalservers erf�llen.
Ein Vergleich verschiedener reiner \ac{CMS}-Systeme ist in der Bachelorarbeit ``Konzept und prototypische Implementierung eines �bergreifenden Dokumenten- und Medienmanagements'' zu finden.
\cite{Bachelorarbeit}

\cite{WikiTypo3}
\subsection{Typo3}
\cite{LobacherTypo3}
\subsection{Typo3 Neos}
\subsection{Joomla}
\subsection{Drupal}
\subsection{Auswertung der m�glichkeiten}
\section{Typo3}
\subsection{Typo3 8.2}
\subsection{Extensions}
\subsection{TypoScript}
\subsection{Fluid}
\section{Modulare Extensions}
\section{Neue Webtechnologien}
\subsection{Google Polymer}
\subsection{AngularJS}
\subsection{HTML5}
\subsection{CSS3}
\subsubsection{Bootstrap}
\subsubsection{Foundation Framework}
\subsection{PHP7}
\subsection{Google Dart}
\section{Zusammenspiel der Technologien}
\section{Zusammenfassung}
\section{Fazit}